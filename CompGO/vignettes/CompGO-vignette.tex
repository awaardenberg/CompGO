\documentclass[11pt, oneside]{article}
\usepackage{geometry}
\geometry{letterpaper}
\usepackage{graphicx}
\usepackage{amssymb}
\usepackage[affil-it]{authblk}

\title{EnrichmentVisualiser vignette}
\author{Sam Bassett}
\affil{Developmental and Stem Cell Biology Lab,\\Victor Chang Cardiac Research Institute,\\Darlinghurst, Sydney, Australia}
\date{}

\begin{document}
\maketitle
\section{Introduction}
This package contains functions to accomplish several tasks. It is able to download full genome databases from UCSC, import .bed files easily, annotate these .bed file regions with genes (plus distance) from aforementioned database dumps, interface with DAVID to create functional annotation and gene ontology enrichment charts based on gene lists (such as those generated from input .bed files) and finally visualise and compare these enrichments using either directed acyclic graphs or scatterplots.

\section{Dumping UCSC databases}
We provide the UCSC mm9 refGene database as a data file, but if a different track is needed this function can be used to do so. It can sometimes take some time to run, since it's downloading a large file.

<<>>=
## Get UCSC database for human genome, refGene track
library(CompGO)
db = ucscDbDump(genome="hg19")
@

\section{.bed file importing and annotation}
Reading a .bed file into R is easy: \begin{verbatim}
## Not run, needs path
bed = read.bed('./path_to_file.bed')
## Sometimes, such as when using biomaRt, the chromosome
## name must be removed, so an option exists to do so
bed = read.bed('./path_to_file.bed', subChr=TRUE)
\end{verbatim}
Once a .bed file is read, you can start annotation. Here we use the sample .bed file provided.
\begin{verbatim}
require('EnrichmentVisualiser')
data(bed.sample)
data(ucsc.mm9)
## This step can take some time
geneList = annotateBedFromUCSC(bedfile=bed.sample,db=ucsc.mm9)
str(geneList)
\end{verbatim}

\section{Functional Enrichment and visualisation}
This gene list can then be passed on to DAVID for gene ontology enrichment analysis. This returns an object of type DAVIDFunctionalAnnotationChart, but it can be accessed similarly to a data.frame.
\begin{verbatim}
## Not run, since it needs a registered email
fnAnot = getFnAnot_genome(geneList$name2, email="your_registered@email.com",
+	idType="REFSEQ_MRNA",listName="sample1")
str(fnAnot)
\end{verbatim}

After obtaining an enrichment table, we can visualise differences between two sets of genes (or two .bed files) using either a pairwise scatterplot or a combination directed acyclic graph (DAG), where each node (representing an individual GO term) is coloured based on which set it came from.
\begin{verbatim}
## Since this requires multiple annotation files, 
## which we have not supplied, the following won't be run.
anot1 = getFnAnot_genome(geneList1, email="email@email.com")
anot2 = getFnAnot_genome(geneList2, email="email@email.com")

## To produce scatterplot:
plotPairwise(anot1, anot2, cutoff = 0.05) # using benjamini as metric
# plotting all GO terms, even if they only appear in one set
plotPairwise(anot1, anot2, plotNA=TRUE, cutoff=0.1) 

## For DAGs:
plotTwoGODags(anot1, anot2) # no relaxation, default colours
plotTwoGODags(anot1, anot2, relaxPvals = TRUE) # with relaxation
\end{verbatim}


\end{document}
