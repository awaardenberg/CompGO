\documentclass[11pt, oneside]{article}
\usepackage[top=3cm, bottom=3cm, left=3cm, right=3cm]{geometry}
\geometry{letterpaper}
\usepackage{graphicx}		
\usepackage{amssymb}
\usepackage[affil-it]{authblk}

\title{Approaches for comparing Gene Ontologies}
\author{Sam Bassett}
\affil{Developmental and Stem Cell Biology Lab,\\Victor Chang Cardiac Research Institute,\\Darlinghurst, Sydney, Australia}
\date{}

\begin{document}
\maketitle
\section*{Abstract}
%revise
Finding a good method for comparing gene ontology (GO) enrichments between lists of genes is an important problem for functional comparison. Here, several tools are presented that accomplish this task, with their respective benefits and downsides examined. An in-house method of comparing enrichment analyses between sets, CompGO, is also developed. CompGO draws methodologies from the existing approaches, but enhances their analytic potential to include comparisons of gene sets. This enables both similarities and differences between sets to be brought forth that would otherwise not be immediately obvious from superficial comparison.

\section*{Introduction}
Gene Ontology was conceived as the "tool for the unification of biology", since it provides a vocabulary for gene function that is applicable to all eukaryotes \cite{GOConsort00}. Genes have terms assigned to them in three broad categories, namely biological process, molecular function and cellular component, and can have varying degrees of specificity - from the very broad, such as "cell", right down to the very specific.\\
Many online tools have emerged in order to help perform both gene ontology annotation and enrichment analysis, then subsequently visualise the results. They all perform similar functions, but each has their own advantages and downsides. One restriction worth mentioning is that the experimental data used to test the tools did not come from an either an Affymetrix or other similar gene chip, which many tools actually require for the first step of analysis; such tools have not been reviewed here.\\
While this process of enrichment analysis is very useful for inferring possible functional differences corresponding to gene expression profiles, the ability to compare gene ontology profiles between different datasets and visualise these differences can give a much clearer idea of functional similarity. Relatively few of the tools currently available online are able to compare enrichments from multiple gene sets, so the R package CompGO was developed in order to streamline and meaningfully visualise these relationships. 
%Here several enrichment analysis tools are compared, along with two methods of visually comparing the results of GO enrichment analysis created in-house.
\section*{Existing tools}
\subsection*{DAVID}
\subsubsection*{Web Interface}
DAVID, coupled with their excellent R package RDAVIDWebService, fulfils most basic functional annotation requirements. It adopts the core strategy of systematically mapping a large number of interesting genes in a list to the associated biological annotation, then statistically highlighting the most enriched annotations \cite{David08}. From this list of annotations, it can subsequently undergo further analysis such as gene functional classification, functional annotation charts and functional annotation tables. It can also provide a pathway view of enriched terms \cite{David08}. When comparing gene lists, however, the web interface falls slightly short. The best method for ontological comparison found was to use one gene list as the test list and one as the background. This produced a list of terms more significantly enriched in the test list, but not the background. The process had to be run again with the gene lists swapped in order to reciprocally determine enrichment.
\subsubsection*{R package - RDAVIDWebService}
The RDAVIDWebService R package bypasses this restriction by giving fine-grain control over the enrichment process. Gene lists are uploaded to the server via R, and raw data such as functional annotation tables are returned. This means that instead of having to reciprocally compare two gene sets with each other as background, the entire GO enrichment can be done against the organism's genome. The package then provides methods for visualisation, most significantly, the ability to generate graphs from GO terms (including their counts and significance). The only downsides of this package are that a knowledge of R is required in order to appreciate its flexibility, and registration is required in order to use it. It forms the base for CompGO, which builds upon its functionality in order to compare sets of genes.
\subsection*{WebGestalt}
WebGestalt is similar to DAVID's web interface, in that it provides an interface to a server which is capable of several different types of analysis. This includes some that DAVID does not provide, such as protein interaction network analysis. It gives finer control over the statistical tests done to produce the enrichment as well. It does not have a corresponding R package, nor is it as extensible as DAVID. The results produced are similar, but seem to be pruned at a significance level not accessible by the user.
\subsection*{GOEAST}
GOEAST adopts the same core strategy as discussed previously with some extra choice in terms of advanced statistical parameters - it is the only tool that includes Alexa's improved weighting algorithm, a statistical test that is designed to correct for additional (incorrect) overrepresentation of neighbouring GO terms introduced by the hierarchical-dependent relationships of said terms \cite{GOEast08}. The process of GO enrichment using GOEAST is very slow compared to the other tools examined here, instead of taking several seconds to process a list of ~1500 genes GOEAST can take up to 20 minutes. This is a significant barrier to entry, especially when compared to the other tools.\\
The benefit of using GOEAST is their comparison tools. A program called Multi-GOEAST is provided on the website which allows up to three GOEAST result files to be uploaded and compared with one another, where they can be visualised as a DAG with node colours corresponding to amount of enrichment in each of the three sets. However, the user does not have much control over this visualisation process, and the resultant graphs are heavily pruned when compared with DAVID.
\subsection*{clusterProfiler}
ClusterProfiler is an R package with no associated web interface which permits both functional enrichment and visualisation. The first thing to note is that the documentation is woeful. Some things just don't work, necessitating a dive into the R source (on GitHub, mind you, since most of the R functions themselves are obfuscated) in order to figure out how to use them. The package was used initially because it has a nice way of visualising \emph{n} gene sets at once, the percentage enrichment of each term and their \emph{p} values. However, in practice, this visualisation technique only worked for data generated by the package itself (so DAVID results cannot be visualised in this way, for example) and either showed only the five broadest terms or an impossibly long list of terms that don't really show anything meaningful.\\
The function provided would be incredibly helpful for comparing gene lists if it worked correctly, since it also allows the user to provide the comparison function. However, this isn't currently feasible.

\section*{Comparison Table}
\begin{tabular}{ | p{5cm} | p{5cm} | p{5cm} |}
\hline
Name & Pros & Cons \\
\hline
DAVID Web Interface & Great for generating single annotations, does everything in one place.\newline Fast and easy to use, frequently cited.\newline Input genes can be in a variety of different formats. & Not designed for comparing multiple gene lists.\\
\hline
DAVIDWebService R package & Easily extensible, everything can be adjusted.\newline Lends itself to novel analysis.\newline Same flexibility of input genes. & Requires registration, knowledge of R.\newline Queries can sometimes be slow.\newline Each list must be uploaded to the webserver before proceeding.\\
\hline
WebGestalt & Extra features, such as further control over statistics and the ability to generate protein interaction networks.\newline Similar flexibility of input gene set & Also not designed to compare multiple gene lists.\\
\hline
GOEAST & Further statistical control such as Alexa's algorithm\newline Specific GO comparison tool & \emph{Very} slow.\newline Only MGI IDs can be used as input, so generally preprocessing is required before submitting a query.\\
\hline
clusterProfiler & Potentially very useful functions.\newline Simple, requires only a gene list and it will do the annotation without further input. & Functions don't work as documented.\newline Inflexible, only EntrezGene IDs can be used as input.\newline Functions aren't editable, so the benefits of using R are lessened.\\
\hline

\end{tabular}

\section*{CompGO}
All of the currently available tools have limitations when it comes to comparing gene sets. Whether they don't offer much control over the comparison process (GOEAST) or don't natively support comparison at all (the other tools discussed above), it's clear that a method for gene set comparison is currently lacking. To this end, we present CompGO. CompGO is an R pipeline built on DAVID's web service which provides both gene annotation from .bed file coordinates and, importantly, several methods for gene ontology enrichment comparison. First, this method allows much finer control over the data generation and visualisation when compared to the web interfaces; it is also open-source, so it can be easily modified or extended. This is already a real improvement over the existing methods, but in addition, CompGO contains two methods of visualisation that aid the interpretation of data generated.\\
First, a plot of $p$ values from two different enrichment tables can be performed. Each GO term shared by both tables is plotted based on its $p$ value, and the statistical correlation between the two is computed. The Jaccard coefficient of each GO term, which represents the proportion of shared genes compared with the overall set of genes, can also be plotted as colour on the same graph; this gives additional power to the analysis as terms which may be skewed by a few extraneous genes can easily be picked out. Fine control over this plot also is given in the form of function parameters.\\
The other method, adapted from code sourced in the RDAVIDWebService package, is plotting a DAG where individual nodes (representing GO terms) are given colour based on the gene set from which they came. Again, the plotting function is flexible, and this approach aids in seeing specifically which enriched terms come from which set, and which ones are common to both.

\section*{Conclusion}
The ability to compare gene ontologies between different experiments or organisms is both useful and currently nascent. We present a package, CompGO, which builds upon current approaches used by DAVID to perform enrichment analysis on single gene sets and extends them to compare gene sets pairwise, then visually compare the results. 
%In our own analysis, this allowed ...


\begin{thebibliography}{}

\bibitem{GOConsort00}
	The Gene Ontology Consortium, 
	\emph{Gene Ontology: tool for the unification of biology}.
	Nature Genetics, volume 25
	May 2000.

\bibitem{David08}
	Huang, D.W., Sherman, B. \& Lempicki, R.,
	\emph{Systematic and integrative analysis of large gene lists using DAVID bioinformatics resources}.
	Nature Protocols, volume 4
	December 2008.

\bibitem{GOEast08}
	Zheng, Q., Wang, X.,
	\emph{GOEAST: a web-based software toolkit for Gene Ontology enrichment analysis}.
	Nucleic Acids Research, volume 36
	May 2008.

\end{thebibliography}

\end{document}  